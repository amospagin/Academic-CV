%!TEX TS-program = xelatex
%!TEX encoding = UTF-8 Unicode

\documentclass[10pt]{article}

% Packages
\usepackage[margin=1.1in]{geometry}               
\geometry{a4paper}          
\usepackage{eulervm}         		
\usepackage{fontspec}
\usepackage{parskip}  
\usepackage{graphicx}
\usepackage{amssymb}
\usepackage[dvipsnames, svgnames]{xcolor}
\usepackage{academicons}
\usepackage{fontawesome}
\usepackage[colorlinks=true, urlcolor=DarkBlue]{hyperref}
\usepackage{url}
\usepackage{hanging}
\usepackage{fancyhdr}
\usepackage{csquotes}

% Formatting
%\setlength{\parskip}{\baselineskip}

% Fonts
\defaultfontfeatures{Mapping=tex-text,Ligatures=TeX}
\setromanfont[Mapping={tex-text}, 
 	Numbers={OldStyle},
 	Ligatures={Common}]{Minion Pro}
\setmonofont[Scale=0.78]{Pragmata Pro Mono}
\setsansfont{Myriad Pro Semibold}
\newfontfamily\sansfont{Myriad Pro Semibold}

% Colors
\definecolor{grayrule}{HTML}{c6c6c6} % Gray rule

% Header and footer
\setlength{\headheight}{15.2pt}
\pagestyle{fancy}
\fancyhf{}
\lhead{Amos Pagin}
%\chead{Curriculum vitae}
\rhead{Revised: \today}
\cfoot{\thepage}

% Commands
\newcommand{\RR}{\textcolor{grayrule}{\rule[3mm]{\textwidth}{0.5pt}}} 
\newcommand{\oneline}{\\[1\baselineskip]}
\newcommand{\twoline}{\\[2\baselineskip]}

% Document
\begin{document}
\thispagestyle{empty}
\begin{center}
\textsc{} \\[\baselineskip]
%\addfontfeature{LetterSpace=1.5} % Adds tracking for headers
%{\fontsize{18}{48} \selectfont \textsc{curriculum vitae}} \\[1.25\baselineskip]
{\fontsize{18}{48} \selectfont Amos Pagin} \vspace{1.35\baselineskip}
%{\fontsize{22}{48} \selectfont \textsc{amos pagin}} \vspace{1.35\baselineskip}
\end{center}


\begin{minipage}[t]{0.7\linewidth}
Department of Psychology,  \\ 
University of Gothenburg, \\
  PO Box 500,  \\
 405 30 Gothenburg, Sweden\\ \\
  %\vspace{0.75\baselineskip}
 \end{minipage}
 \begin{minipage}[t]{0.5\linewidth}
 %Strålgatan 19\\
 %112 63 Stockholm, Sweden \\
 E-mail: \href{mailto: amos.pagin@psy.gu.se}{amos.pagin@psy.gu.se}\\
 Web: \href{https://www.amospagin.com}{www.amospagin.com} \\ \\
 %Twitter: \href{https://twitter.com/amospagin?lang=en}{\texttt{@amospagin}} 
Revised: \today
 \end{minipage}

%% % Personal details
%\textsc{\Large areas of interest and specialization}  \\
%\vspace{-.3cm}\RR 
%
%Cognitive aging, reserve and resilience, neuroplasticity, neurodegenerative diseases, psychometrics. \vspace{\baselineskip}

% % Personal details
%\textsc{\Large personal details}  \\
%\vspace{-.3cm}\RR 
%
%\begin{minipage}[t]{0.15\linewidth}
%Full name: \\ 
%Date of birth: \\ 
%Place of birth: \\
%\end{minipage}
%\begin{minipage}[t]{0.9\linewidth}
%Amos Michael Joseph Pagin\\
%October 8, 1987 \\
%Stockholm, Sweden \\
%
%\end{minipage}


% Education
%\textsc{\Large education}  \\
{\large Education} \\
\vspace{-.3cm}\RR 

\begin{minipage}[t]{0.075\linewidth}
2025 \\ \\
2020 \\ 
2018
\end{minipage} 
\begin{minipage}[t]{0.92\linewidth}P.hD. Psychology, University of Gothenburg (expected).\\ 
{\footnotesize Dissertation advisors: Prof. Martin Lövdén, Dr. Gaia Olivo.} \\
M.Sc. Social science (psychology track), Uppsala University. \\
%{\footnotesize  Main field of study: Psychology.} \\
\normalsize B.Sc. Psychology, Stockholm University. \\
%{\footnotesize Minors in theoretical philosophy and linguistics.}\\
\end{minipage}

% Publications
%\textsc{\Large publications} \\
{\large Publications} \\
\vspace{-.3cm}\RR 

\textbf{Journal articles}

\begin{hangparas}{0.4in}{1}
\vspace{-0.6em}
[2] Lövdén, M., \textbf{Pagin, A.}, Bartrés-Faz, D., Boraxbekk, C-J., Brandmaier, A. M., Demnitz, N., Drevon, C.A., Ebmeier, K. P., Fjell, A. M., Ghisletta, P., Gorbach, T., Lindenberger, U., Plachti, A., Walhovd, K. B., \& Nyberg, L. (2023). No moderating influence of education on the association between changes in hippocampus volume and memory performance in aging. \emph{Aging Brain, 4}. %\href{https://doi.org/10.0000/3mp7y-537}{https://doi.org/10.0000/3mp7y-537}

[1] \textbf{Pagin, A.} (2019). Exploring the conjunction fallacy in probability judgment: Conversational implicature or nested sets?. \emph{Journal of European Psychology Students, 10}(2). \\
\end{hangparas}

% Grants and fellowships
%\textsc{\Large grants and fellowships}\\
{\large Grants and fellowships} \\
\vspace{-.3cm}\RR 

\textbf{Grants obtained in competition (as principal applicant)}

\vspace{-0.6em}
\begin{minipage}[t]{0.075\linewidth}
2022 
\end{minipage} 
\begin{minipage}[t]{0.92\linewidth}Grant from the Nordic Mensa Fund. 5 000 EUR. Awarded for the research project \enquote{Learning and problem solving}. \vspace{\baselineskip}
\end{minipage}


 
%% Departmental talks
%\textsc{\Large departmental talks}\\
%\vspace{-.3cm}\RR 
%
%\begin{minipage}[t]{0.075\linewidth}
%2021 
%\end{minipage} 
%\begin{minipage}[t]{0.92\linewidth}Learning and transfer in relational reasoning: Grounding the Flynn effect in socioculturally acquired cognitive skill. Lifespan Development Lab, Department of Psychology, University of Gothenburg. April 26th. \vspace{\baselineskip}
%\end{minipage}

% Teaching experience
%\textsc{\Large teaching experience}  \\
{\large Teaching experience} \\
\vspace{-.3cm}\RR 

%\textbf{At the University of Gothenburg} \\
\textbf{Psychology: Intermediate course (Module: Method)} \\
Department of Psychology, University of Gothenburg. \\
Undergraduate course (Swedish). \\Advised and graded quantitative student projects. \\Terms taught: Spring 2023, Autumn 2023, Spring 2024.  \oneline
\textbf{Cognitive psychology} \\
Department of Psychology, University of Gothenburg. \\
Master's level course (English). \\Lectures and seminars on working memory, long-term memory, and the psychology of language. \\
Terms taught: Spring 2021, Spring 2022, Spring 2023, Spring 2024.  \oneline
\textbf{Psychology II (Module: Perception and cognition)} \\
Department of Neuroscience and Physiology, University of Gothenburg. \\
Undergraduate course (Swedish). \\ Lectures on working memory and long-term memory.\\
Terms taught: Fall 2021, Fall 2022. \oneline
%\textbf{At Stockholm University} \\
\textbf{Neuroscience, cognition, and learning II} \\
Department of Psychology, Stockholm University.\\
Undergraduate course (Swedish). \\Lecture and seminar on introductory statistics and SPSS.\\
Terms taught: Spring 2017.\oneline
\textbf{Psychology III (Module: Methods and statistics)} \\
Department of Psychology, Stockholm University. \\
Undergraduate course (Swedish). \\ Lectures and seminars on statistics (descriptives, t-tests, ANOVA, linear regression, power). \\
Terms taught: Spring 2017, Fall 2017, Spring 2018. \\


% Academic supervision
%\textsc{\Large academic supervision}  \\
{\large Academic supervision} \\
\vspace{-.3cm}\RR 

\begin{minipage}[t]{0.075\linewidth}
2018 
\end{minipage} 
\begin{minipage}[t]{0.92\linewidth}
Co-advisor, Bachelor's thesis in Psychology, Clara Rosenthal. Department of Psychology, Stockholm University. \vspace{\baselineskip}
\end{minipage}

% University service
%\textsc{\Large departmental/university service} \\
{\large Departmental/university service} \\
\vspace{-.3cm}\RR 
\textbf{}


\begin{hangparas}{0.4in}{1}
Representative to the Faculty of Social Sciences’ Doctoral Student Council. Department of Psychology, University of Gothenburg. Fall 2022.

Organizer of the Lifespan Development Lab weekly seminar series. Department of Psychology, University of Gothenburg. Fall 2021.

%\vspace{\baselineskip}
\end{hangparas}

 \vspace{\baselineskip}



% Skills
%\textsc{\Large computer skills}\\
%\vspace{-.3cm}\RR 

%R, LaTeX, SPSS \\

% Languages
%\textsc{\Large languages}\\
{\large Languages} \\
\vspace{-.3cm}\RR 

Swedish (native), English (fluent)
 \vspace{\baselineskip}


{\large References} \\
\vspace{-.3cm}\RR 

Upon request.

\end{document}  
